\subsection{Пер. Уллу-Кёль Восточный} 
Перевал Уллу-Кёль Восточный расположен в северо-восточном отроге Даутского хребта и связывает д.~р. Кичкинакол Уллу-Кёльский с д.~р. Махар. В самом этом разветвлении, между хребтом и отрогом, находится каскад из двух озёр: собственно оз. Уллу-Кёль (карач.-балк. <<Большое озеро>>) и оз. Гитче-Кёль (карач.-балк. <<маленькое озеро>>). Озеро Уллу-Кёль, по описаниям, исключительно живописно и является значимой достопримечательностью района --- однако единственный доступный для походов 1 к.~с. выход из цирка, в котором оно располагается, --- это пер. Уллу-Кёль Восточный, который с технической точки зрения был самым сложным среди всех перевалов нашего маршрута, и который, по идее, нельзя рекомендовать в качестве первого. 

Альтернативой перевалу Уллу-Кёль Восточный может служить пер. Уллу-Кёль Нижний (н/к), через который в д.~р. Махар можно попасть от нижнего озера, Гитче-Кёль, --- именно этим маршрутом прошёл в 2018 г. Андрей Королёв, и именно этот вариант был знаком руководителю. Замена однако при этом получается совершенно неравнозначной: с одной стороны, Уллу-Кёль Нижний идеально выполняет функцию акклиматизационного перевала, а с другой стороны, при проходе через него теряется возможность увидеть верхнее озеро, Уллу-Кёль, --- если только не идти на него радиально. В связи с этим, при планировании маршрута вставал вопрос о том, стоит ли рассматривать маршрут через оз. Уллу-Кёль и пер. Уллу-Кёль Восточный, или же следует удовлетвориться видом оз. Гитче-Кёль и пройти через Уллу-Кёль Нижний. В конечном итоге руководитель пошёл на риск и проложил маршрут через Уллу-Кёль Восточный, назначив вариант через Уллу-Кёль Нижний запасным: существенную роль в этом сыграло нежелание полностью копировать машрут Королёва, а также сильная нелюбовь руководителя к радиальным выходам. 

Перевал ориентирован с ССЗ на ЮЮВ. На перевальном взлёте со стороны озера большую часть времени лежит снег, сам взлёт в верхней своей части достигает крутизны 40--50\degree~и заканчивается снежным карнизом. Вероятность, что группа в итоге пойдёт на этот перевал, руководителем оценивалась примерно в 30\%; ставку при этом планировалось делать: 1) на благоприятную ориентацию склона со стороны подъёма; 2) на то, что во второй половине августа снега будет меньше, чем это фиксировалось в среднем в отчётах; 3) на то, что все участники группы будут оснащены кошками. 