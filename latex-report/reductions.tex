\section*{Сокращения, используемые в отчёте}
\addcontentsline{toc}{section}{Сокращения, используемые в отчёте}
\begin{tabular}{p{0.08\textwidth} p{0.7\textwidth}}
	МКК                                  &   Маршрутно-квалификационная комиссия  \\
	ФСТ                                &   Федерация спортивного туризма  \\
	к.с.                               &   категория сложности (похода)  \\
	ст.с.							& степень сложности (похода) \\
	н/к                            &   некатегорированный (перевал, препятствие) \\
	орогр.                &   орографически  \\
	ЧХВ                          &   чистое ходовое время  \\
	т/б                         &   туристическая база \\
	т/к                         &   туристический клуб \\
	а/л                  &   альпинистский лагерь \\
		с. & село \\
	г. & город \\
	ст. & станция (канатной дороги) \\
	верш.               &   вершина \\
	пер.               &   перевал \\
	оз.             &   озеро \\
	р.             &   река \\
	д.	&	долина\\
	хр. &   хребет \\
	тр. &   травянистый \\
	ос. &   осыпной \\
	ск. &   скальный \\
	сн. &   снежный \\
	лед. &   ледовый \\
	лев. &   левый \\
	пр. &   правый \\
	пхд	&	по ходу движения	\\
	рук. &   руководитель \\	
	КЧР & Карачаево-Черкесская республика\\
	КБР & Кабардино-Балкарская республика\\
	ГКХ	&	Главный Кавказский хребет \\
	м.н. & место ночёвки
\end{tabular}
\clearpage