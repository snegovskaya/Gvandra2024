\section{Организация и проведение похода}
\subsection{Цели и задачи маршрута. Подготовка, выбор маршрута}
Даша, необходимо твоё творчество. 

\textcolor{teal}{Окэй. Итак...} 

При места проведения маршрута в целом, я как руководитель опиралась на следующие соображения: 
\begin{enumerate} 
	\item Транспортная доступность горного района и стоимость трансфера. 

	Поскольку группа, как я и надеялась, практически полностью состояла из новичков, в этом случае особенно важно было затратить минимум сил и средств на логистику, --- т.~е. выбрать район с наилучшей транспортной доступностью за наименьшие деньги. При такой постановке задачи почти автоматически отсеиваются районы дальнего, и ближнего зарубежья --- расположенные, например, в Киргизии --- а также вся азиатская часть России, как, например, Алтай, --- и дальнейший выбор сводится фактически к одному из районов Кавказа. 
	
	\item Концентрированность препятствий 
	Дополнительным фактором в сторону выбора Кавказа послужило также и то, что в отличие, например, от Алтая, для этих гор характерны довольно короткие долины, поэтому подход к перевалам занимает, как правило, один день и позволяет поддерживать интерес группы на приемлемом уровне \textcolor{teal}{(нас это, правда, не спасло, хех)}. 
	
	\item Разнообразие рельефа  
	С методической точки зрения, а также, опять-таки, для поддержания интереса группы, хотелось продемонстрировать участникам как можно больше разнообразных типов рельефа: в частности, осыпной в диапазоне от крупных до мелких осыпей, и, самое главное, снег и лёд. В связи с этим \textcolor{teal}{(так-таки почему только западный рассматривали-то?)} среди всех доступных районов Западного Кавказа выбор падал на Гвандру как на наиболее высокий район, с достаточным количеством снега и льда. 
	
	\item Эффект кульминации 
	У меня как у руководителя было глубокое убеждение, что первый поход должен обладать понятным, с позволения сказать, сюжетом и иметь свою кульминацию — и в нашем случае движение с запада на восток с постепенно открывающимися видами на Эльбрус как на главную доминанту Кавказа и собственно проход по его ледовым полям представляли из себя очевидный сюжет с очевидной же кульминацией. По моим предположениям, это должно было положительно сказаться на восприятие группой  маршрута в целом. 
	
	\item Знакомые локации 
	Фактор, который формально не был в списке определяющих критериев, но по факту являлся таковым, — это то, что спланированный маршрут был практически полной копией маршрута, который я как участник проходила под руководством Королёва А.~Э. в 2018 г. \cite{Korolyov2018}. При этом при планировании своего маршрута мне хотелось по возможности отойти от маршрута Андрея и не копировать его точь-в-точь, однако по результатам изучения отчётов прошлых лет становится, в общем, понятно, что альтернатив пройденным в 2018 году перевалам немного, и в каждом отроге ГКХ, который планируется пересекать в ходе такого маршрута, существует 1–2 перевала категории 1А, которые имеет смысл идти в походе первой категории сложности.
\end{enumerate} 

\subsection{Изменение маршрута и их причины}
Маршрут пройден без изменений.
\subsection{Развёрнутый график движения}
\begin{table}[h!]
	\centering
	\resizebox{0.9\textwidth}{!}{%
	\begin{tabular}{|>{\centering\arraybackslash}m{0.045\linewidth}
								|>{\centering\arraybackslash}m{0.02\linewidth}
								|>{\centering\arraybackslash}m{0.43\linewidth}
								|>{\centering\arraybackslash}m{0.04\linewidth}
								|>{\centering\arraybackslash}m{0.07\linewidth}
								|>{\centering\arraybackslash}m{0.05\linewidth}
								|>{\centering\arraybackslash}m{0.043\linewidth}
								|>{\centering\arraybackslash}m{0.085\linewidth}|}
		\hline						
		Дата	&	\begin{turn}{90}День\end{turn}	&	Участок маршрута	&	\begin{turn}{90}Км с $k=1.2$\end{turn}	&	Набор /сброс, м	&	ЧХВ	&	\begin{turn}{90}Высота ночёвки, м\end{turn}	&	Способы передвижения	\\
		\hline
		
		18.08	&	1	&	г.~Минеральные воды~--- аул Верхний Учкулан~--- д.р Учкулан~--- д.р. Кичкинакол Уллукёльский	&	хзхз	&	$+454$\newline$-0$	& 4:88	&	2000	&	Машина,\newline Пешком	\\
		\hline
		19.08	&	2	&	д.р. Кичкинакол Уллукёльский~--- оз. Гитче-Кёль~--- оз. Уллу-Кёль 	&	7,2	& $+700$\newline$-0$		& 5:20		& 2700		&	Пешком	\\
		\hline
		20.08	&	3	&	м.н.~--- \textbf{пер. Уллу-Кёль Восточный (1А$^\star$, 3050)}~--- кош в д.р. Трёхозёрная~--- д.р. Махар	&	4	&5		&6		&7		&	Пешком	\\
		\hline
		21.08	&	4	&	м.н.~--- т/б <<Глобус>>~--- д.р. Гондарай~--- д.р. Джалпаккол	&	4	&5		&6		&7		&	Пешком	\\
		\hline
		22.08	&	5	&	м.н.~--- д.р. Кичкинекол Джалпаккольский~--- м.н. под моренным валом пер. Джалпаккол Северный	&	4	&5		&6		&7		&	Пешком	\\
		\hline
		23.08	&	6	&	м.н.~--- \textbf{пер. Джалпаккол Северный (1А$^\star$, 3411)}~--- зелёные ночёвки на спуске в д.р. Мырды	&	4	&5		&6		&7		&	Пешком	\\
		\hline
		24.08	&	7	&	м.н.~--- д.р. Мырды~--- а/л <<Узункол>>	&	4	&5		&6		&7		&	Пешком	\\
		\hline
		25.08	&	8	&	м.н.~--- д.р. Кичкинекол~--- д.р. Таллычат~--- Поляна Крокусов	&	4	&5		&6		&7		&	Пешком	\\
		\hline
		26.08	&	9	&	м.н.~--- \textbf{пер. Кичкинекол Малый (1А, 3206)}~--- д.р. Чунгур-Джар	&	4	&5		&6		&7		&	Пешком	\\
		\hline
		27.08	&	10	&	м.н.~--- \textbf{пер. Перемётный (1А, 3255)}~--- д.р. Танышхан	&	4	&5		&6		&7		&	Пешком	\\
		\hline
		28.08	&	11	&	м.н.~--- д.р. Чиринкол~--- д.р. Кубань &	4	&5		&6		&7		&	Пешком	\\
		\hline
		29.08	&	12	&	м.н.~--- погранзастава <<Хурзук>>~(рад.)~--- д.р. Уллу-Кам	&	4	&5		&6		&7		&	Пешком	\\
		\hline
		30.08	&	13	&	м.н.~--- \textbf{пер. Хотютау (1А$^\star$, 3546)}~--- лед. Большой Азау~--- оз. Эльбрусское~--- ст. <<Старый Кругозор>>~--- поляна Азау&	4	&5		&6		&7		&	Пешком, Канатная дорога	\\
		\hline
	\end{tabular}
	}

\end{table}



\newpage