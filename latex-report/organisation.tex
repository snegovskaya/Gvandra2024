\section{Организация и проведение похода}
\subsection{Цели и задачи маршрута. Выбор нитки маршрута}
При места проведения маршрута в целом, я как руководитель, опиралась на следующие соображения: 
\begin{enumerate} 
	\item \textbf{Транспортная доступность горного района и стоимость трансфера.}
	Поскольку группа, как и планировалось, практически полностью состояла из новичков, в этом случае особенно важно было затратить минимум сил и средств на логистику, --- т.~е. выбрать район с наилучшей транспортной доступностью за наименьшие деньги. При такой постановке задачи почти автоматически отсеиваются районы дальнего и ближнего зарубежья --- расположенные, например, в Кыргызстане, --- а также вся азиатская часть России, как, например, Алтай, --- и дальнейший выбор сводится, фактически, к одному из районов Кавказа. 
	
	\item \textbf{Концентрированность препятствий.}
	Дополнительным фактором в сторону выбора Кавказа послужило также и то, что в отличие, например, от Алтая, для этих гор характерны довольно короткие долины, поэтому подход к перевалам занимает, как правило, один день и позволяет поддерживать интерес группы на приемлемом уровне.
	
	\item \textbf{Разнообразие рельефа.} 
	С методической точки зрения, а также, опять-таки, для поддержания интереса группы, хотелось продемонстрировать участникам как можно больше разнообразных типов рельефа: в частности, осыпи в диапазоне от крупных до мелких и, самое главное, снег и лёд. В связи с этим среди всех доступных районов Западного Кавказа выбор падал на Гвандру как на наиболее высокий район, с достаточным количеством снега и льда. 
	
	\item \textbf{Эффект кульминации.}
	У меня как у руководителя было глубокое убеждение, что первый поход должен обладать понятным, с позволения сказать, сюжетом и иметь свою кульминацию~--- и в нашем случае движение с запада на восток с постепенно открывающимися видами на Эльбрус как на главную доминанту Кавказа и, собственно, проход по его ледовым полям представляли из себя очевидный сюжет с очевидной же кульминацией. По моим предположениям, это должно было положительно сказаться на восприятие группой  маршрута в целом. 
	
	\item\textbf{ Знакомые локации.}
	Фактор, который формально не был в списке определяющих критериев, но по факту являлся таковым, — это то, что спланированный маршрут был практически полной копией маршрута, который я как участник проходила под руководством Королёва~А.Э. в 2018 г. \cite{Korolyov2018}. При этом при планировании своего маршрута мне хотелось по возможности отойти от маршрута Андрея и не копировать его точь-в-точь, однако по результатам изучения отчётов прошлых лет становится, в общем, понятно, что альтернатив пройденным в 2018 году перевалам немного, и в каждом отроге ГКХ, который планируется пересекать в ходе такого маршрута, существует 1–2 перевала категории 1А, которые имеет смысл идти в первом походе первой категории сложности.
\end{enumerate} 
\subsection{Логистика}
Подъезд осуществлён на поезде 033М Москва--Владикавказ до станции Минеральные Воды (прибытие в 03:40). Стоимость проезда на август 2024 г. составляла 7800~\faRub, купе (обратно – 4700~\faRub, плацкарт). От Минеральных Вод до аула Верхний Учкулан (время в пути 4 часа) добирались на трансфере, заказанном через Саракуева Бориса (89289503868, 89298843175,  \href{mailto: bezonec@list.ru}{bezonec@list.ru}). Стоимость трансфера трансфера туда составила 18000~\faRub, обратно (от поляны Азау) — 15000~\faRub. Стоимости доставки забросок в т/б Глобус и а/л Узункол составили 4000~\faRub~и 6000~\faRub.
Коллективный пропуск в пограничную зону КЧР был оформлен за 4 месяца до начала похода через электронную почту пограничного управления ФСБ по КЧР~--- \href{mailto: pu.kcherkes@fsb.ru}{pu.kcherkes@fsb.ru} и отправлен письмом по указанному адресу. Пропуск в КБР не требуется, так как пер. Хотютау в 2023 году был исключён из пограничной зоны \cite{order_kbr}.
\subsection{Аварийные выходы из маршрута и его запасные варианты}
\textbf{Аварийными выходами} с маршрута являлись:
\begin{itemize}
	\item На первом этапе: спуск к т/б <<Глобус>>;
	\item На втором этапе: спуск к а/л <<Узункол>>;
	\item На третьем этапе: спуск к погранзаставе <<Актюбе>> (Хурзук).
\end{itemize}
\textbf{Запасными вариантами} маршрута являлись:
\begin{itemize}
	\item Замена пер. Уллу-Кёль Восточный (1А$^\star$, 3050) на пер. \textbf{Уллу-Кёль Нижний (н/к, 2933)};
	\item Отказ от пер. Перемётный (1А, 3255), спуск по д.р. Чунгур-Джар;
	\item Отказ от пер. Хотютау (1А$^\star$), спуск по д.р. Кубань к погранзаставе <<Хурзук>>
\end{itemize}
\subsection{Изменение маршрута и их причины}
Маршрут пройден без изменений.
\subsection{Обеспечение безопасности на маршруте}
Группа была зарегистрирована в региональных отделениях МЧС по КЧР и КБР (две заявки, оформленные на сайте МЧС за две недели до похода).
Адреса и реквизиты для связи с региональными органами МЧС:
\begin{itemize}
	\item \textbf{ГУ МЧС России по КЧР:} 369000, г. Черкесск, ул. Кавказская, д. 33.\\
	Тел.: +7(878) 226-60-56 (по тургруппам), +7(878) 226-62-00 (дежурный);
	
	\item \textbf{ГУ МЧС России по КБР:} 360017, г. Нальчик, ул. Чернышевского, д. 19.\\
	Тел.: +7(866)274-36-03 (по тургруппам), +7(866) 387-14-89 (дежурный);	
\end{itemize}
Для регулярного обмена сообщениями, отслеживания положения группы на карте, а также возможности экстренной связи, в группе имелся спутниковый треккер IRIDIUM Rockstar 360. Стоимость аренды треккера в <<Альпиндустрии>> на 15--21 день составила 7100~\faRub, залог~--- 50000~\faRub. Нам повезло попасть на демострационный период тарифа треккера, в связи с чем все сообщения были для нас безлимитны и бесплатны. Предварительное тестирование треккера в Москве показало, что спутниковые сигналы в столице эффективно глушатся: сообщения приходили не чаще раза в сутки. В походе с приёмом и отправкой сообщений и координат на сервер проблем не возникало, среднее время отправки составляло 30 минут.
На участников группы было оформлено два страховых полиса (по 6 фамилий в каждом) компании Евроинс для
занятий самодеятельным и спортивным туризмом и горным треккингом до высоты 3500~м. Страховка обеспечивала проведение поисково-спасательных работ и транспортировку вертолётом, сумма покрытия 50000~\faEur. Стоимость страхового полиса на 13 дней составила 5734~\faRub~на человека.

\subsection{Перечень наиболее интересных природных и исторических объектов, занятий на маршруте}
\begin{enumerate}[noitemsep,topsep=0pt,parsep=0pt,partopsep=0pt]
	\item Каскад озёр Уллу-Кёль и Гитче-Кёль в д.р. Кичкинакол Уллукёльский; 
	\item Долина реки. Мырды (карач.-балк. <<Болото>>)~--- считается одной из самых красивых долин Гвандры; 
	\item Нарзанные источники у т/б <<Глобус>>; 
	\item Ледник Чунгур-Джар; 
	\item Большой Кавказский хребет; 
	\item Руины поселений в верховьях р. Кубань; 
	\item Эльбрус и виды на него с разных перевалов;
	\item Хычины.
\end{enumerate}

\paragraph{Темы практических занятий:}

\begin{itemize}
	\item Техника передвижения по травянисто-осыпным склонам;
	\item Техника передвижения по снегу, льду.
\end{itemize}

\newpage
\subsection{Развёрнутый график движения}
\begin{table}[h!]
	\centering
	\resizebox{0.95\textwidth}{!}{%
		\begin{tabular}{|>{\centering\arraybackslash}m{0.045\linewidth}
				|>{\centering\arraybackslash}m{0.02\linewidth}
				|>{\centering\arraybackslash}m{0.43\linewidth}
				|>{\centering\arraybackslash}m{0.09\linewidth}
				|>{\centering\arraybackslash}m{0.1\linewidth}
				|>{\centering\arraybackslash}m{0.05\linewidth}
				|>{\centering\arraybackslash}m{0.09\linewidth}
				|>{\centering\arraybackslash}m{0.13\linewidth}|}
			\hline						
			Дата	&	\begin{turn}{90}День\end{turn}	&	Участок маршрута	&	Км с $k=1.2$	&	Набор /сброс, м	&	ЧХВ	&	Высота ночёвки, м	&	Способы передвижения	\\
			\hline
			
			18.08	&	1	&	г.~Минеральные воды~--- аул Верхний Учкулан~--- д.р Учкулан~--- д.р. Кичкинакол Уллукёльский	&	5.3	&	$+650$\newline$-0$	& 2:46	&	2200	&	Машина,\newline Пешком	\\
			\hline
			19.08	&	2	&	д.р. Кичкинакол Уллукёльский~--- оз. Гитче-Кёль~--- оз. Уллу-Кёль 	&	5.6	& $+650$\newline$-0$		& 3:25		& 2850		&	Пешком	\\
			\hline
			20.08	&	3	&	м.н.~--- \textbf{пер. Уллу-Кёль Восточный (1А$^\star$, 3050)}~--- кош в д.р. Трёхозёрная~--- д.р. Махар	&	7.2	& $+200$\newline$-1190$		& 7:39	& 1860		&	Пешком	\\
			\hline
			21.08	&	4	&	м.н.~--- т/б <<Глобус>>~--- д.р. Гондарай~--- д.р. Джалпаккол	&	11.3	&$+390$\newline$-225$		& 3:54		& 2120		&	Пешком	\\
			\hline
			22.08	&	5	&	м.н.~--- д.р. Кичкинекол Джалпаккольский~--- м.н. под моренным валом пер. Джалпаккол Северный	&	5.8	& $+620$\newline$-0$		& 3:56	& 2740		&	Пешком	\\
			\hline
			23.08	&	6	&	м.н.~--- \textbf{пер. Джалпаккол Северный (1А$^\star$, 3411)}~--- зелёные ночёвки на спуске в д.р. Мырды	&	5.0 	& $+660$\newline$-395$		& 6:16		& 3015		&	Пешком	\\
			\hline
			24.08	&	7	&	м.н.~--- д.р. Мырды~--- а/л <<Узункол>>	&	7.5	& $+0$\newline$-960$		& 3:53		& 2060		&	Пешком	\\
			\hline
			25.08	&	8	&	м.н.~--- д.р. Кичкинекол~--- д.р. Таллычат~--- Поляна Крокусов	&	7.1	& $+780$\newline$-0$		& 3:23		& 2840		&	Пешком	\\
			\hline
			26.08	&	9	&	м.н.~--- \textbf{пер. Кичкинекол Малый (1А, 3206)}~--- д.р. Чунгур-Джар	&	4.6	& $+360$\newline$-520$		& 2:42		& 2680		&	Пешком	\\
			\hline
			27.08	&	10	&	м.н.~--- \textbf{пер. Перемётный (1А, 3255)}~--- д.р. Танышхан	&	7.1	& $+575$\newline$-935$		& 6:50		& 2320		&	Пешком	\\
			\hline
			28.08	&	11	&	м.н.~--- д.р. Чиринкол~--- д.р. Кубань &	12.7	& $+90$\newline$-500$		& 3:23		& 1890		&	Пешком	\\
			\hline
			29.08	&	12	&	м.н.~--- погранзастава <<Хурзук>>~(рад.)~--- д.р. Уллу-Кам	&	20.9	& $+1210$\newline$-370$		& 7:15		& 2725		&	Пешком	\\
			\hline
			30.08	&	13	&	м.н.~--- \textbf{пер. Хотютау (1А$^\star$, 3546)}~--- лед. Большой Азау~--- оз. Эльбрусское~--- ст. <<Старый Кругозор>>~--- поляна Азау & 10.9	& $+800$\newline$-615$		& 4:25		& 2915		&	Пешком, Канатная дорога	\\
			\hline
			\multicolumn{3}{|c|}{\textbf{\textit{\Large{Итого:}}}} & \large{\textbf{111.0}} & \large{$\mathbf{+6985}$\newline$\mathbf{-5210}$	}	& \multicolumn{3}{c|}{\large{\textbf{58:08}\newline\textbf{2д 10ч 08мин}}} \\
			\hline
		\end{tabular}
}	
	
\end{table}



\clearpage