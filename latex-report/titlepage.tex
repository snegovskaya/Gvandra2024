\begin{titlepage}

	\begin{center}
		РОО <<Федерация спортивного туризма Московской области>>\\
		ОО г. Долгопрудного <<Федерация спортивного туризма>>\\
	\end{center}

	
	
	\begin{center}
		\Large{\bfseries{ОТЧЁТ}} \\
		\normalsize о прохождении горного спортивного туристического маршрута первой категории сложности по Западному Кавказу (Гвандра), совершённом с 18 по 30 августа 2024 г. группой туристов Горной секции МФТИ ФСТ Московской области, г. Долгопрудный
	\end{center}
	\vspace{1.8 cm}
	
	\parskip \textbf{Маршрутная книжка:} 03.03.01, прикладные математика и физика \\ 
	\textbf{Направленность подготовки:} Физическая и квантовая электроника
	\vspace{3 cm}
	
	\null\hfill
	\begin{minipage}{0.5\textwidth}
		\begin{flushleft} \large
			\textbf{Студент:} \\
			Остапив Алексей Юрьевич
			\vspace{0.5 cm}
			\hrule
			\vspace{-0.6 em}
			\begin{center}
				\small \textit{(подпись студента)} \\
			\end{center}
			
			\vspace{-0.4 em}
			\large
			\textbf{Научный руководитель:} \\
			Коняшкин Алексей Викторович \\
			к.ф.-м.н., старший научный сотрудник ФИРЭ им. В.А. Котельникова РАН 
			\vspace{0.5 cm}
			\hrule
			\vspace{-0.6 em}
			\begin{center}
				\small \textit{(подпись научного руководителя)} \\
			\end{center}
			
		\end{flushleft}
	\end{minipage}
	
	% \vspace{4.3cm}
	\vfill
	\vfill
	\begin{center}
		Москва   \the\year{}
	\end{center}
	
\end{titlepage}