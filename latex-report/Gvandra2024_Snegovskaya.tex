\documentclass[a4paper, 12pt]{report}
\usepackage[utf8]{inputenc} % выбор кодировки кода
\usepackage[T1, T2A]{fontenc} % выбор внутренней кодировки 
\usepackage[english, russian]{babel} % выбор языка
% \righthyphenmin = 2 % минимальное число букв после переноса: может пригодиться
\usepackage{amsmath} % русский текст в формулах
\usepackage{color} % цвет текста
\usepackage{ulem} % для зачёркивания текста

\usepackage[left=15mm, right=10mm, top=20mm, bottom=20mm]{geometry} % поля
\usepackage{indentfirst} % отступ первого абзаца
\setlength{\parindent}{1,25cm} % длина отступа первой строки абзаца
\linespread{1,5} % междустрочный интервал

\usepackage{titlesec} % настройка заголовков
\renewcommand{\thesection}{\arabic{section}} % противотараканная мера для report'а: делаю section заголовком первого уровня
\titleformat{\section}{\centering \large \bfseries}{\thesection}{1ex}{\MakeUppercase}{}
\titleformat{\subsection}{\centering \large \bfseries}{\thesubsection}{1ex}{}{}
\titleformat{\subparagraph}[runin]{\bfseries}{\thesubparagraph}{}{}{}
% \titleformat{\bibliography}{\centering \large \bfseries}{\thebibliography}{}{}{} % попытка настроить заголовок для списка литературы

\usepackage{graphicx}
\graphicspath{{./Photos/}}


\usepackage{comment} % Для многострочных комментариев
\usepackage{xcolor} % Для девочек 

\begin{document}
\section{Таблица маршрута} 
dfТут я перекатаю плоды Лёшиного труда
\section{Участники} 
Тут будет список наших \sout{долбанов} зайчиков.
\section{Концепт (типа abstract)} 
Тут будет что-то про красоток природы и про то, какая я в целом продуманная молодец.
\section{Литературный обзор}

\subsection{пер. Перемётный} 
Пер. Перемётный представляет из себя одно из ключевых и, с точки зрения тактики прохождения, одно из самых сложных препятствий на маршруте. На сайте Вестры \cite{WestraCat} имеется всего пять упоминаний об этом перевале, одно из которых — фотоальбом вершин и перевалов Кавказа за авторством \textcolor{teal}{какого-то чувака, на которого мне сейчас лень ссылаться}, и ещё одно — отчёт Королёва А.~Э. 2018 г. \cite{Korolyov2018}, в котором автор принимал участие, и в ходе которого было решено отказаться от сквозного прохождения Перемётного, заменив его радиальным. \textcolor{teal}{(Попросить Андрея напомнить о причинах!)} Что касается остальных трёх источников, то это отчёты Зеленцовой Е.~А. 2000 г. \cite{Zelentsova2000}, Истягиной Е.~Е. 2015 г. и Анучиной С. 2019 г. \cite{Anuchina2019}: в первых двух пер. Перемётный берётся с запада на восток, из д.~р. Чунгур-джар в д.~р. Талычхан, а в третьем --- с востока на запад, из д.~р. Талычхан в д.~р. Чунгур-джар. 

Основную сложность при прохождении перевала с запада на восток представляет спуск: как в отчёте Зеленцовой, так и в отчёте Истягиной описаны сложности, с которыми сталкивается группа как при планировании спуска, так и в процессе. Обоим руководителям пришлось забирать влево по ходу движения, обходя каньон ручья и мощные скальные выходы, но даже при этом группы Зеленцовой и Истягиной попадали в ловушку из труднопроходимых зарослей рододендрона и берёзового криволесья. Впрочем, после спуска оба автора указывают на возможный оптимальный вариант спуска, который пролегает ещё левее (севернее), в обход криволесья. 

Их выводы полностью подтверждает отчёт Анучиной С. от 2019 г., которая вела группу из трёх человек по каменистому распадку --- руслу пересохшего ручья --- в обход берёзового криволесья. Судя по описанию, этот вариант выглядит довольно щадящим и представляет из себя в основном путь по курумнику с небольшим участком рододендроновых зарослей наверху и высокой травы у подножия склона. В том же отчёте утверждается, что по левому берегу р. Талычхан проходит тропа, которая сливается с тропой, ведущей по левому берегу р. Чунгур-джар. 

Идея взять Перемётный возникла в качестве альтернативы прохождению очень неприятного участка вдоль Чунгурджара: крутого спуска вдоль водопада по заросшему рододендронами склону. Технически на этом склоне имеется тропа, но, по моим воспоминаниям, она легко терялась, и в целом склон был усеян небольшими камнями размером с ботинок, которые невозможно было разглядеть сквозь рододендроновые заросли, и на которых ничего не стоило подвернуть ногу. В случае же с Перемётным мы будем иметь тоже не самый приятный для группы, но более предсказуемый и оттого безопасный спуск по курумнику и обещанную тропу, которая проходит по травянистой долине без большого сброса. Таким образом, я решаю разменять гарантированно неприятный и небезопасный спуск на, предположительно, более долгий, но чуть более безопасный спуск; предположительно терпимую тропу и взятие нового нетривиального перевала с исследованием предположительно оптимального по нему маршрута.

\textcolor{teal}{1. Прописать тайминги по всем трём отчётам (плюс мои прикидки по реверснутому прохождению маршрута Анучиной); 2. Дописаться до Анучиной; 3. Вставить фоточки.}

\bibliography{Gvandra2024_Snegovskaya.bib}
\bibliographystyle{plain}

\end{document}