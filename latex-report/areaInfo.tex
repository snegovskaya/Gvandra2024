\section{Общегеографическая и туристическая характеристика района}

\subsection{Географическое положение и туристские особенности района}
Тут всякие красивые слова про всякое. Могу накатать сам, Можешь  сама, Даш. 


\subsection{Варианты подъезда и отъезда}
Подъезд осуществлён на поезде 033М Москва--Владикавказ до станции Минеральные Воды (прибытие в 03:40) Стоимость проезда на август 2024 г. составляла 7800~\faRub, купе (обратно – 4700~\faRub, плацкарт). От Минеральных Вод до аула Верхний Учкулан (время в пути 4 часа) добирались на трансфере, заказанном через Саракуева Бориса (89289503868, 89298843175,  \href{mailto: bezonec@list.ru}{bezonec@list.ru}). Стоимость трансфера трансфера туда составила 18000~\faRub, обратно (от поляны Азау) — 15000~\faRub. Стоимости доставки забросок в т/б Глобус и а/л Узункол составили 4000~\faRub  и 6000~\faRub.

\subsection{Аварийные выходы из маршрута и его запасные варианты}
\textbf{Аварийными выходами} с маршрута являлись:
\begin{itemize}
	\item На первом этапе: спуск к т/б <<Глобус>>;
	\item На втором этапе: спуск к а/л <<Узункол>>;
	\item На третьем этапе: спуск к погранзаставе <<Хурзук>>
\end{itemize}


\textbf{Запасными вариантами} маршрута являлись:
\begin{itemize}
	\item Замена пер. Уллу-Кёль Восточный (1А$^\star$, 3050) на пер. \textbf{Уллу-Кёль Нижний (н/к, 2933)};
	\item Отказ от пер. Перемётный (1А, 3255), спуск по д.р. Чунгур-Джар;
	\item Отказ от пер. Хотютау (1А$^\star$), спуск по д.р. Кубань к погранзаставе <<Хурзук>>
\end{itemize}


\subsection{Характеристика средств передвижения, особенности погодных условий}

До места старта~--- аула Верхний Учкулан мы добрались на машине, далее весь маршрут был пройден пешком до финиша~--- ст. Кругозор.

\subsection{Расположение пограничных зон, заповедников, порядок получения пропусков, дислокация ПСО, медучреждений и другие полезные данные}

Группа была зарегистрирована в региональных отделениях МЧС по КЧР и КБР (две заявки, оформленные на сайте МЧС за 2 недели до похода).
Адреса и реквизиты для связи с региональными органами МЧС:
\begin{itemize}
	\item \textbf{ГУ МЧС России по КЧР:} 369000, г. Черкесск, ул. Кавказская, д. 33.
	Тел.: +7(878) 226-60-56 (по тургруппам), +7(878) 226-62-00 (дежурный);
	
	\item \textbf{ГУ МЧС России по КБР:} 360017, г. Нальчик, ул. Чернышевского, д. 19
	Тел.: +7(866)274-36-03 (по тургруппам), +7(866) 387-14-89 (дежурный);
	
\end{itemize}
Коллективный пропуск в пограничную зону КЧР был оформлен за 4 месяца до начала похода через электронную почту пограничного управления ФСБ по КЧР~--- \href{mailto: pu.kcherkes@fsb.ru}{pu.kcherkes@fsb.ru} и отправлен письмом по указанному адресу. Пропуск в КБР не требуется, так как пер. Хотютау в 2023 году был исключён из пограничной зоны \cite{order_kbr}.

\subsection{Перечень наиболее интересных природных и исторических объектов, занятий на маршруте}
что то про красивые озёра и виды на Эльбрус

Темы практических занятий:
\begin{itemize}
	\item Техника передвижения по травянисто-осыпным склонам;
	\item Техника передвижения по снегу, льду.
\end{itemize}



\newpage