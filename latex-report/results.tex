\section{Итоги похода, выводы и рекомендации по совершённому походу (от лица руководителя)}

\subsection{Подготовка к походу и взаимодействие с группой} 
	\begin{enumerate}
		\item При подготовке участников особое внимание требуется уделить навыку подстраховки альпенштоком. В нашем случае ни у кого из участников не было проблем с движением по курумнику --- что можно считать большим везением --- и лимитирующим фактором для скорости группы становились именно крутые спуски, на которых большинство участников не смогло обеспечить себе уверенную страховку, потому что навык спуска с альпенштоком не был достаточно отработан; 
		\item Для некоторых участников хождение ночью с фонариком оказалось большим стрессом. Этот навык тоже следует отрабатывать отдельно на разного рода ночных ориентированиях при подготовке к походу; % \st{В смысле, не ходить ночью? Не, фигня!}
		\item Долгие утренние сборы в походе были организационной ошибкой руководителя. Предположительно, сборы можно было ускорить, причём, по отзывам участников, без особых усилий с их стороны; 
		\item В походе, в начале каждого дня участникам не хватало озвучивания плана на день, а непосредственно на марше~--- коммуникации руководителя с участниками; 
		\item При подготовке хотелось бы большей автоматизации при расчёте весов и учёте снаряжения --- работу с гугл-доком следует оптимизировать.
	\end{enumerate} 
	
\subsection{Выбор и прохождение маршрута}  
	\begin{enumerate}
		\item Решение провести поход в конце августа я оцениваю в конечном итоге как разумное, хотя не все ожидания оправдались. А именно, с точки зрения погоды на маршруте --- по моим личным ощущениям ---конец августа не сильно выигрывает у конца июля -- начала августа: также не редки грозы и дожди. Тем не менее, благодаря поздним срокам была достаточная уверенность в том, что лед. Большой Азау окажется открытым --- и эти ожидания оправдались --- а также скорее повезло с состоянием снега на пер. Уллу-Кёль Восточный. По информации, полученной нами от руководителя похода 1 к.с., Антона Колотова, в том же районе, но в июле, на Уллу-Кёле Восточном может быть снежный карниз.
		\item Упомянутый выше перевал Уллу-Кёль Восточный при том количестве снега, которое было в нашем случае, и при тех условиях, на мой взгляд, проходим, но только при условии наличия кошек у всей группы. Также вызывает вопрос наилучшая техника его прохождения: стоит ли в конце подъёма ставить кошку боком на всю ступню или идти на три такта, если снег раскис; 
		\item Насчёт кошек: моё мнение было и остаётся таким, что на данном маршруте кошки строго необходимо иметь всем участникам группы из-за перевалов. Уллу-Кёля Восточного и Джалпаккола Северного; 
		\item Спуск с пер. Перемётного на 900 м по высоте вместо 300 был явной ошибкой руководителя и штурмана; 
		\item Тем не менее, само прохождение пер. Перемётного и спуск в д.~р. Чиринкол по левому берегу р. Таллычат я оцениваю скорее как удачное решение; 
		\item В целом, пройденная нами нитка маршрута, за исключением, возможно, пер. Уллу-Кёль Восточный, видится мне эталонной и отвечает всем заявленным требованиям для горного маршрута 1 к.~с., особенно, если для многих участников этот поход является первым. Пер. Уллу-Кёль Восточный, возможно, имеет смысл заменить на пер. Уллу-Кёль Нижний, но до конца я в этом не уверена. 
	\end{enumerate}